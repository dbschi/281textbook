\section{Span and Linear Independence}
Throughout this section, let $\mathbb{K}$ be a field. We constrain ourselves to work in $V$, a $\mathbb{K}$-vector space.
\definition{Span}{
	Let $k$ be some positive integer. Let  $\{v_1, v_2, ... , v_k\}\subseteq V$. The \textbf{span} of $\{v_1, v_2, ... , v_k\}$ is denoted as \[
	\textrm{span}(v_1,v_2,...,v_k)
	\]
	and is the \textit{smallest} subspace of $V$ containing $\{v_1, v_2, ... , v_k\}$.
}
The \textit{smallest} here means that if another subspace $W$ contains $\{v_1,...,v_k\}$, $W$ cannot be a subset of $\textrm{span}(v_1,...,v_k)$. How do we know such a subspace exists? We can take the intersection of all the subspaces containing $\{v_1,...,v_k\}$ \[
\textrm{span}(v_1,...,v_k) = \bigcap_{W \textrm{ subspace containing } \{v_1,...,v_k\}} W
\]
which is a subspace containing $\{v_1,...,v_k\}$ and is a subset of all other subspaces containing $\{v_1,...,v_k\}$. We know $V\subseteq V$ is a subspace containing $\{v_1,...,v_k\}$, so the intersection between at least one set and is thus well-defined.
\proposition{
     The span of $\{v_1, v_2, ... , v_k\}\subseteq V$ is 
    all linear combinations of the vectors in the set:
    \[\textrm{span}\{v_1, v_2, ... , v_k\} = \{c_1v_1 + c_2v_2 + ... + c_kv_k | c_1, c_2, ... c_k \in \mathbb{K}\}\]
}
\begin{proof}
	We first show $\textrm{span}\{v_1, v_2, ... , v_k\} \supseteq \{c_1v_1 + c_2v_2 + ... + c_kv_k | c_1, c_2, ... c_k \in \mathbb{K}\}$. That is, for every $W$ subset containing $v_1,...,v_k$, $W$ must also contain $c_1v_1+...+c_kv_k$. \\
	We now show $\textrm{span}\{v_1, v_2, ... , v_k\} \subseteq \{c_1v_1 + c_2v_2 + ... + c_kv_k | c_1, c_2, ... c_k \in \mathbb{K}\}$. The right side is a set that nonempty, is closed under addition and scalar multiplication, and contains $v_1= 1v_1+0v_2+...+0v_k$,...,$v_k=0v_1+...+0v_{k-1}+1v_k$. Therefore, it is one of the $W$ subspaces whose intersection is used to construct the span.
\end{proof}

% insert examples including what the span of one, two, etc. vectors is

It can be difficult to how to think about what the span of a set of vectors looks like, though it is also important to develop
an intuition for it as more complex techniques are developed. It is also important to consider what vectors span a given subspace.

\example{For example, the vector space of $\mathbb{R}^3$ is spanned by the set of unit vectors $\{i, j, k\}$.}

Vector spaces' sizes can be compared by the minimum number of vectors required to span them. This "dimensionality" will be used more
later, but for now it is important that vector spaces can be either finite or infinite dimensional.

\example{Vector spaces like $\mathbb{R}^n$ can be spanned by $n$ vectors and thus are finite dimensional. Vector
spaces that are composed of less linear objects, like the set of all functions $\mathbb{R} \rightarrow \mathbb{R}$, are often
infinite dimensional. In this case, note that there's no finite list of functions such that all other functions from $\mathbb{R}
\rightarrow \mathbb{R}$ are linear combinations of them.}

