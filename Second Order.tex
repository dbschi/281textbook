\chapter{Second Order Differential Equations}
\setcounter{exercisecounter}{0}

\setcounter{thmcounter}{1}
\section{Introduction}
\noindent Our study of differential equations continues with second order equations.
Such equations are defined by their highest order derivative of the function in question being a second derivative.
They are of the form
\begin{align*}
    a(t)y'' + b(t)y' + c(t)y = f(t),
\end{align*}
where $a(t)$, $b(t)$, $c(t)$, and $f(t)$ are functions of $t$.
We will study methods to solve these equations, both homogeneous and inhomogeneous, and discuss the existence and uniqueness of solutions.
\section{Constant Coefficients}
The first technique we will study in solving second order differential equations is for cases of homogeneous equations with constant coefficients. Such equations are of the form 
\[ay'' + by' + cy = 0\]
This equation suggests we are looking for solutions $y(t)$ for which the derivatives can be easily summed together to produce zero. Methods in calculus suggests the solution
\[y(t) = e^{rt}\]
Let's suppose this is the case, then
\[y'(t) = re^{rt}\]
\[y''(t) = r^2e^{rt}\]
Substituting these into our differential equation
\[a(r^2e^{rt}) + b(re^{rt}) + c(e^{rt}) = 0\]
\[(ar^2 + br + c)e^{rt} = 0\]
Since $e^{rt} \neq 0$, this implies we want to find values $r$ which satisfy 
\[ar^2 + br + c = 0\]
The fundamental theorem of algebra states that solving this equation will produce at least one complex root and 2 roots total counted for multiplicity. In this section, we will look at this case in which the equation produces two roots distinct $r_1, r_2 \in \mathbb{R}$. Thus we get two solutions, 
\[y_1 = e^{r_1t}\]
\[y_2 = e^{r_2t}\]
We check linear independence with the Wronskian,
\[\begin{vmatrix}
e^{r_1t} & e^{r_2t} \\
r_1e^{r_1t} & r_2e^{r_2t} \\
\end{vmatrix} = r_2e^{r_1}e^{r_2} - r_1e^{r_1t}e^{r_2t} = (r_2-r_1)e^{(r_1+r_2)t} \neq 0 \]
since the exponential function is never zero and $r_1, r_2$ are distinct.
This gives us two linearly independent solutions that produce a basis for the set of solutions to this differential equation, thus our general solution is
\[y = c_1e^{r_1t} + c_2e^{r_2t}\]
where $c_1, c_2$ are undetermined coefficients to be determined by initial conditions. 
\exercises
\begin{exerciselist}
\item Solve the following homogeneous second-order differential equation with the given initial conditions:
\begin{align*}
    y'' - 3y' + 2y = 0, \\
    y(0) = 1, \quad y'(0) = 0. \\
\end{align*} 
\item Solve the following homogeneous second-order differential equation with the given initial conditions:
\begin{align*}
    y'' + 4y' + 4y = 0, \\
    y(0) = 0, \quad y'(0) = 1.\\
\end{align*}
\end{exerciselist}
\section{Complex Roots}
\noindent We now look at cases of equations in the previous section for which the characteristic equation produces complex roots. However, a quick remark is needed first. 
\definition{Make this a remark somehow}{A polynomial of degree 2 with real coefficients can either have, 2 real, 2 complex, 1 repeated real or 1 repeated complex roots.}
This means that any degree two polynomial cannot have one real root and one complex root. We will now look at cases for which we have two complex roots. 
Suppose we have a differential equation of the form
\[ay'' + by' + cy = 0\]
The previous section suggests we solve the quadratic equation
\[ar^2 + br+ c = 0\] 
to find $r_1$ and $r_2$ that produce solutions $y_1 = e^{r_1t}$ and $y_2 = e^{r_2t}$ to the differential equation. If
\[r_1 = a_1 + b_1i\]
\[r_2 = a_2 + b_2i\]
then our general solution becomes
\[y = c_1e^{(a_1+b_1i)t} + c_2e^{(a_2+b_2i)t}\]
However, certain cases prove it useful to find real solutions. 
In these cases we use Euler's Identity 
\[e^{(a+bi)t} = e^{at}(cos(bt) + isin(bt))\].
The power in this technique is that it produces two real solutions from a single complex solution. We will prove this now.
\proof{
Suppose $y = u + iv$ is a complex solution to the second order homogeneous differential equation with constant coefficients 
\[ay''+by'+cy=0\]
where $u$ and $v$ are real valued functions.
We have
\[y' = u' + iv'\]
\[y''=u''+iv''\]
Substituting into our differential equation
\[a(u''+iv'')+b(u'+iv')+c(u+iv)=0\]
\[= (au''+bu'+u)+i(av''+bv'+cv)=0\]
This suggests both the real and imaginary parts of this equation must be zero thus we have,
\[au''+bu'+cu =0\]
\[i(av''+bv'+cv)=0\]
This results suggests that both $u$ and $v$ are real solutions to the differential equation. 
}
Now if we let $u = cos(bt)$ and $v = sin(bt)$ we have obtained two real solutions to our differential equation from one complex solution. We still must have the $e^at$ factor multiplied by $u+v$ and our two undetermined coefficients to be satisfied by initial conditions; therefore our general solution is
\[y = e^{at}(c_1cos(bt) + c_2sin(bt))\]
We check linear independence with the Wronskian
\[\begin{vmatrix}
    
\end{vmatrix}\]
From this result we see that for cases of complex roots only one root suffices to obtain a general solution. 

\section{Method of Reduction of Order}
When our characteristic equations of second-order constant coefficient homogeneous equations results in repeated roots, we obtain only one solution. Therefore we look to develop a technique to find a second solution. We suggest a solution of the form 
\[y(t) = v(t)y_1(t)\]
where $y_1(t)$ is the first solution found. 
Taking derivatives we have
\[y'(t) = v'(t)y_1(t) + v(t)y'_1(t)\]
\[y''(t) = v''(t)y_1(t) + 2v'(t)y'_1(t) + v(t)y''_1(t)\]
Substituting this into the differential equation to solve for the undetermined equation $v(t)$
\[ay''+by'+cy = 0 \]
\[a(v''y_1 + 2v'y'_1 + vy''_1) + b(v'y_1 + v'y'_1) + c(vy_1) = 0 \]
\[av''y_1 + v'(2ay'_1+by_1) + v(ay''_1 + by'_1 + cy_1) = 0\]
The third term is zero since $y_1$ is a solution to that differential equation which is the same as what we started out with. Therefore we have
\[av''y_1 + v'(2ay'_1+by_1)=0\]
\[\frac{v''}{v'} = \frac{-(2ay'_1+by_1)}{ay_1}\]
We can solve this separable differential equation for $v'$
\[\int \frac{dv}{v} = \int (-\frac{2y'_1}{y_1} - \frac{b}{a})dt\]
This gives us the solution
\[v' =\frac{1}{y_1^2}Ce^{-\int \frac{b}{a}dt}\]
We have kept the argument in the exponential in integral form as this method is generalizable to any differential equation in which we have one solution and require another for a general solution, however, in the case of constant coefficients the exponential will be $e^{bt/a}$.
\exercises
\begin{exerciselist}
\item Given that \( y_1(x) = e^x \) is a solution to the differential equation:
\[y'' - y' = 0,\]
use the method of reduction of order to find a second, linearly independent solution \( y_2(x) \).
\item Suppose \( y_1(x) = x \) is a solution to the differential equation:
\[x^2 y'' - 3x y' + 3y = 0 \quad \text{for } x > 0.\]
Use the method of reduction of order to find a second solution \( y_2(x) \) that is linearly independent of \( y_1(x) \).
\end{exerciselist}
\section{Variation of Parameters}
Thus far we have developed techniques to solving homogeneous second order equations. We now turn our attention to finding methods to solve inhomogeneous equations.
\[ay''+by'+cy = f(t)\]
We find solutions to inhomogeneous equations by adding a general solution to the corresponding inhomogeneous equation with a particular solution to the inhomogeneous equation. 
\[y = y_g + y_p\]
Suppose we can find two linearly independent solutions to the corresponding homogeneous equation of (), $y_1, y_2$. The method of variation of parameters suggests we look for a particular solution of the form
\[y(t)_p = u(t)y_1(t)+u_2(t)y_2(t)\]
where $y_1(t), y_2(t)$ are solutions to the corresponding homogeneous equation and $u_1(t), u_2(t)$ are undetermined coefficients. 
Taking the derivative
\[y' = u'_1y_1+u_1y'_1 + u'_2y_2+u_2y'_2\]
These calculations are made simpler if we set 
\[u'_1y_1 + u'_2y_2 = 0 \]
Therefore $y'$ becomes
\[y' = u_1y'_1 + u_2y'_2 \]
Finding $y''$
\[y'' = u'_1y'_1 + u_1y''_1 + u'_2y'_2+u_2y''_2\]
Substituting this into our differential equation
\[a(u'_1y'_1 + u_1y''_1 + u'_2y'_2+u_2y''_2)+ b(u_1y'_1 + u_2y'_2 )+c(u_1y_1 + u_2y_2) = f(t)\]
\[u_1(ay''_1 + by'_1+cy_1) + u_2(ay''_2 + by'_2+cy_2) + au'_1y'_1+au'_2y'_2 = f(t) \]
Since $y_1$ and $y_2$ are solutions to the corresponding homogeneous equation, the first two terms are zero. Thus, our equation reduces to
\[au'_1y'_1+au'_2y'_2 = f(t)\]
We now have two equations and two unknowns.
\[u'_1y_1 + u'_2y_2 = 0 \]
\[au'_1y'_1+au'_2y'_2 = f(t)\]
From this we obtain
\[u'_1 = \frac{-y_2f(t)/a}{y_1y'_2 - y'_1y_2}\]
\[u'_2 = \frac{y_1f(t)/a}{y_1y'_2 - y'_1y_2}\]
We can integrate to find $u_1$ and $u_2$.
\[u_1 = \int \frac{-y_2f(t)/a}{y_1y'_2 - y'_1y_2} dt\]
\[u_2 = \int \frac{y_1f(t)/a}{y_1y'_2 - y'_1y_2} dt \]
You may notice that the argument in the denominator is the Wronksian thereby implying that if our solutions $y_1, y_2$ are not linearly independent, then when don't have the requisite information to form a general solution to the differential equation. It that case we must return to section 3.5 and find a second linearly independent equation via Method of Reduction of Order. 
\linebreak
\linebreak
This provides us with our particular solution to the inhomogeneous differential equation
\[y_p = y_1\int \frac{-y_2f(t)/a}{y_1y'_2 - y'_1y_2} dt + y_2\int \frac{y_1f(t)/a}{y_1y'_2 - y'_1y_2} dx\]
We add this to the general solution to the corresponding homogeneous equation to obtain the full solution to the inhomogeneous equation.
\[y(x) = y_g + y_p\]
\example{Find the general solution to the differential equation
\[y'' - 3y' + 2y = 2e^x\]
}
The characteristic equation of the corresponding homogeneous equation is
\[r^2 - 3r + 2 = 0\]
This equation has roots $r_1 = 1$ and $r_2 = 2$. Thus the general solution to the corresponding homogeneous equation is
\[y_g = c_1e^x + c_2e^{2x}\]
We can use equation () compute the particular solution to the differential equation.
\[y_1 = e^x\]
\[y_2 = e^{2x}\]
\[y_1' = e^x\]
\[y_2' = 2e^{2x}\]
\[y_p = e^x\int \frac{-e^{2x}2e^x}{e^x2e^{2x} - e^{2x}e^x}dx + e^{2x}\int \frac{e^x2e^x}{e^x2e^{2x} - e^{2x}e^x}dx\]
\[= e^x\int \frac{-2e^{3x}}{e^x2e^{2x} - 2e^{3x}}dx + e^{2x}\int \frac{2e^{2x}}{e^x2e^{2x} - 2e^{3x}}dx\]
\[= e^x\int \frac{-2e^{3x}}{2e^{3x}}dx + e^{2x}\int \frac{2e^{2x}}{2e^{3x}}dx\]
\[= e^x\int -1dx + e^{2x}\int e^{-x}dx\]
\[= -e^x + e^{2x}(-e^{-x})\]
\[= -e^x - e^x\]
\[= -2e^x\]
Thus our particular solution is
\[y_p = -2e^x\]
Therefore our full solution to the differential equation is
\[y(x) = c_1e^x + c_2e^{2x} - 2e^x.\]
\exercises
\begin{exerciselist}
\item Solve the non-homogeneous differential equation:
\[y'' + y = \sin(x),\]
using the method of variation of parameters. Assume that the complementary solution is given by:
\[y_c(x) = C_1 \cos(x) + C_2 \sin(x).\]
\item Use the method of variation of parameters to solve the non-homogeneous differential equation:
\[x^2 y'' - 3x y' + 3y = x^3,\]
for \( x > 0 \). Assume that the complementary solution is:
\[y_c(x) = C_1 x + C_2 x^3.\]
\end{exerciselist}
\section{Method of Undetermined Coefficients}
\noindent There are certain classes of inhomogeneous equations such that we can propose a solution form and algebraically solve for specifying parameters. Such classes usually involve equations of constant coefficients and inhomogeneous terms of familiar functions like exponentials or sinusoidals. As in the previous section, to find a full solution to an inhomogeneous equation we sum together a general solution to the corresponding homogeneous equation with a particular solution to the inhomogeneous equation. Suppose we have the second order inhomogeneous differential equation
\[a(t)y'' + b(t)y'+c(t)y = Acos(\omega t) + Bsin(\omega t)\]
We propose a particular solution of the form
\[y_p = X_1cos(\omega t) + X_2sin(\omega t)\]
Taking derivatives
\[y' = -\omega X_1sin(\omega t) + \omega X_2cos(\omega t)\]
\[y'' = -\omega^2X_1cos(\omega t) - \omega^2X_2sin(\omega t)\]
Substituting into our differential equation
\[a(-\omega^2X_1cos(\omega t) - \omega^2X_2sin(\omega t)) + b( -\omega X_1sin(\omega t) + \omega X_2cos(\omega t) + c( X_1cos(\omega t) + X_2sin(\omega t)) \]
\[= Acos(\omega t) + Bsin(\omega t)\]
We rearrange to get a single cosine and sine term on each side
\[(-a\omega^2X_1+b\omega X_2+cX_1)cos(\omega t) + (-a\omega^2X_2 - b\omega X_1 + cX_2)sin(\omega t)\]
\[= Acos(\omega t) + Bsin(\omega t)\]
From this it is apparent that 
\[(-a\omega^2+c)X_1 + b\omega X_2 = A\]
\[(-a\omega^2+c)X_2 - b\omega X_1 = B\]
We can solve this system of equations to find
\[X_1 = \frac{aB-bA}{a^2\omega^2 + b^2\omega^2 - c}\]
\[X_2 = \frac{aA+bB}{a^2\omega^2 + b^2\omega^2 - c}\]
Thus our particular solution is
\[y_p = \frac{aB-bA}{a^2\omega^2 + b^2\omega^2 - c}cos(\omega t) + \frac{aA+bB}{a^2\omega^2 + b^2\omega^2 - c}sin(\omega t)\]
We add this to the general solution to the corresponding homogeneous equation to obtain the full solution to the inhomogeneous equation.
\[y(x) = y_g + y_p\]
where $y(g)$ will dependent on the form of our differential equation. 
Now suppose we have an inhomogeneous equation of the form
\[a(t)y'' + b(t)y' + c(t)y = e^{\alpha t}\]
We propose a particular solution of the form
\[y_p = Xe^{\alpha t}\]
Taking derivatives
\[y' = \alpha Xe^{\alpha t}\]
\[y'' = \alpha^2Xe^{\alpha t}\]
Substituting into our differential equation
\[a\alpha^2Xe^{\alpha t} + b\alpha Xe^{\alpha t} + cXe^{\alpha t} = e^{\alpha t}\]
\[X(a\alpha^2 + b\alpha + c) = 1\]
\[X = \frac{1}{a\alpha^2 + b\alpha + c}\]
Thus our particular solution is
\[y_p = \frac{1}{a\alpha^2 + b\alpha + c}e^{\alpha t}\]
We add this to the general solution to the corresponding homogeneous equation to obtain the full solution to the inhomogeneous equation.
\[y(x) = y_g + y_p\]
where $y(g)$ will once again be dependent on the form of our differential equation.
\example{Find the general solution to the differential equation
\[y'' + 4y = 2cos(2t)\]
}
Since we have a sinusoidal inhomogeneous term, we propose a particular solution of the form
\[y_p = X_1cos(2t) + X_2sin(2t)\]
Taking derivatives
\[y' = -2X_1sin(2t) + 2X_2cos(2t)\]
\[y'' = -4X_1cos(2t) - 4X_2sin(2t)\]
Substituting into our differential equation
\[-4X_1cos(2t) - 4X_2sin(2t) + 4X_1cos(2t) + 4X_2sin(2t) = 2cos(2t)\]
\[= 2cos(2t)\]
This implies that $X_1 = 1/2$ and $X_2 = 0$. Therefore our particular solution is
\[y_p = \frac{1}{2}cos(2t)\]
The general solution to the corresponding homogeneous equation is
\[y_g = c_1cos(2t) + c_2sin(2t)\]
Thus our general solution to the differential equation is
\[y(x) = c_1cos(2t) + c_2sin(2t) + \frac{1}{2}cos(2t)\]
\example{Find the general solution to the differential equation
\[y'' + 4y = 2e^{2t}\]
}
Since we have an exponential inhomogeneous term, we propose a particular solution of the form
\[y_p = Xe^{2t}\]
Taking derivatives
\[y' = 2Xe^{2t}\]
\[y'' = 4Xe^{2t}\]
Substituting into our differential equation
\[4Xe^{2t} + 4Xe^{2t} = 2e^{2t}\]
\[= 2e^{2t}\]
This implies that $X = 1/4$. Therefore our particular solution is
\[y_p = \frac{1}{4}e^{2t}\]
The general solution to the corresponding homogeneous equation is
\[y_g = c_1cos(2t) + c_2sin(2t)\]
Thus our general solution to the differential equation is
\[y(x) = c_1cos(2t) + c_2sin(2t) + \frac{1}{4}e^{2t}\]
\exercises
\begin{exerciselist}
\item Solve the following non-homogeneous differential equation using the method of undetermined coefficients:
\[y'' - 3y' + 2y = e^{2x},\]
subject to the initial conditions $y(0) = 0$ and $y'(0) = 1$.   
\item Solve the following non-homogeneous differential equation using the method of undetermined coefficients:
\[y'' + y = \sin(x),\]
subject to the initial conditions $y(0) = 1$ and $y'(0) = 0$.
\end{exerciselist}
\section{Existence and Uniqueness}
