\chapter{Span and Linear Independence}

Now that we are practiced with vector spaces, we can begin taking linear combinations of vectors in an arbitrary vector space.

\definition{Span} {
    For some finite set of vectors $\{v_1, v_2, ... , v_k\}$ in a vector space $V$, the \textit{span} of that set is the set of 
    all linear combinations of the vectors in the set: \newline 
    \[span\{v_1, v_2, ... , v_k\} = \{c_1v_1 + c_2v_2 + ... + c_kv_k | c_1, c_2, ... c_k \in \mathbb{R}\}\]
}

Remark: note that the span of an empty set is technically the zero vector: $span\{\empty\} = \{0_V\}$.

\proposition{\textit{The span of any set of vectors in $V$ is always a subspace of $V$}}

\textit{Proof.} The span of a set of vectors includes all linear combinations of them, so taking two linear combinations and 
making more combinations of them will still live in the span. In other words, the span is by definition closed under scalar
multiplication and vector addition.

% insert examples

