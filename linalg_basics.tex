\chapter{Linear Algebra Basics}
    The foundational abstraction of linear algebra is the vector space. A \textit{vector space} is essentially a collection of objects that it
makes sense to take linear combinations of. Two operations must be defined: addition and scalar multiplication. A well defined
vector space meets all of the vector space axioms, which will be listed shortly. Many consequences can be drawn from these axioms,
and we can build up linear algebra to solve any linear problem.

Remark: One has to specify the field of scalars (\textit{field} just means number system) related to the vector space. In most cases
here, we will be talking about real vector spaces (the field of scalars is $\mathbb{R}$). However, vector spaces can be defined with
many other fields of scalars. Examples will follow the definition.

\definition{Vector Space Axioms} {
    A \textit{vector space} $V$ is a set closed under operations of addition ($V + V \rightarrow V$) and scalar multiplication 
    ($\mathbb{K} \cdot V \rightarrow V$) for field of scalars $mathbb{K}$ that obeys the following 8 axioms: \newline
    (VS1) $x + y = y + x$  $\forall x, y \in V$ (commutativity of vector addition) \newline
    (VS2) $x + (y + z) = (x + y) + z$  $\forall x, y, z \in V$ (associativity of vector addition) \newline
    (VS3) There exists some vector $0_V$ s.t. $x + 0_V = x$  $\forall x \in V$ (additive identity / the zero vector) \newline
    (VS4) $\forall x \in V$,  $\exists y \in V$ s.t. $x + y = 0_v$ (additive inverse) \newline
    (VS5) $1 \cdot x = x$  $\forall x \in V$ (multiplicative identity) \newline
    (VS6) $a \cdot (b \cdot x) = (a \cdot b) \cdot x$  $\forall a, b \in \mathbb{K}, x \in V$ (commutativity of scalar multiplication) \newline
    (VS7) $a \cdot (x + y) = a \cdot x + a \cdot y$  $\forall a \in \mathbb{K}, x, y \in V$ \newline
    (distributivity of scalar multiplication over vector addition) \newline
    (VS8) $(a + b) \cdot x = a \cdot x + b \cdot x$  $\forall a, b \in \mathbb{K}, x \in V$ \newline 
    (distributivity of scalar multiplication over scalar addition) \newline
}

$\mathbb{R}^{n}$ is the most intuitive example of a vector space, and it is easy to check that all of the axioms hold for it. Note
that many other properties also hold for $\mathbb{R}^n$. The reason these properties are not axioms is because they are derivable
from the axioms (Examples?)

Definition: If $V$ is a vector space and $W \subseteq V$, then $W$ is a \textit{subspace} of $V$ if and only if $W$ is a vector space
in its own right (it's closed under addition and scalar multiplication). Note that you do not have to recheck the axioms for $W$,
since you know they hold for $V$ and $W \subseteq V$.

Remark: Returning to the idea of vector spaces over fields other than the real numbers, the best example is $\mathbb{C}$ (the complex numbers).
It is a vector space over the field of real numbers, where it is isomorphic to $\mathbb{R}^2$, but it is also a vector space over the complex
numbers (a complex vector space), where is is isomorphic to $\mathbb{R}^1$. Another example is $\mathbb{Q}^1 \subset \mathbb{R}^1$. This is
not a subspace, since it's not closed under scalar multiplication by real numbers, but it is a vector space  when using the field of rational
numbers, since it is then closed under scalar multiplication.

Moving forward, it can be assumed that any vector space is over the field of real numbers, unless otherwise noted.

