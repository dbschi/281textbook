\chapter{Linear Algebra Basics}

\setcounter{exercisecounter}{0}

\setcounter{thmcounter}{1}
    The foundational abstraction of linear algebra is the vector space. A \textit{vector space} is essentially a collection of objects that it
makes sense to take linear combinations of. Two operations must be defined: addition and scalar multiplication. A well defined
vector space meets all of the vector space axioms, which will be listed shortly. Many consequences can be drawn from these axioms,
and we can build up linear algebra to solve any linear problem.

Remark: One has to specify the field of scalars (\textit{field} just means number system) related to the vector space. In most cases
here, we will be talking about real vector spaces (the field of scalars is $\mathbb{R}$). However, vector spaces can be defined with
many other fields of scalars. Examples will follow the definition.
%\iffalse
\subsection*{Where is this going?}
In the previous chapter, we have seen vector manipulation in $\reals^n$. This chapter tries to generalize some observations we have about $\reals^n$ to other mathematical objects.
For now, we call these sets that have `$\reals^n$'-vector-like properties \textbf{vector spaces}. This notion is a bit abstract, and we usually default to $\real^n$
to gain intuition about vector spaces and motivate new definitions and techniques. It just so happens that \textit{every finite dimensional real vector space is `almost the same' as $\reals^n$},
so many conclusions we get from $\reals^n$ naturally extend to other vector spaces.
%\fi

In the previous chapter, \href{prop:1.7}{one proposition} lists $8$ characteristics of $\reals^n$. Let us extract all these $8$
statements, and define a vector space to be a set that satisfies these properties.
\definition{Vector Space}{
	Let $\mathbb{K}$ be a field, and $V$ be a set closed under operations addition $+ : V\times V\to V$ and multiplication $\cdot : \mathbb{K} \times V \to V$. We call $V$ a \textbf{vector space}, or a \textbf{$\mathbb{K}$-vector space} to specify the field if the following axioms hold. \\
%	For all $\vec{u},\vec{v},\vec{w} \in V$, and $a,b \in \mathbb{K}$. 
	\begin{enumerate}
		\item \textit{(Associativity)}$x + (y + z) = (x + y) + z$  $\forall x, y, z \in V$.
		\item \textit{(Commutativity)} $x + y = y + x$  $\forall x, y \in V$.
		\item \textit{(Identity)} There exists some vector $0_V$ s.t. $x + 0_V = x$  $\forall x \in V$.
		\item \textit{(Inverse)} $\forall x \in V$,  $\exists y \in V$ s.t. $x + y = 0_v$.
		\item \textit{(Scalar multiplication)} $a \cdot (b \cdot x) = (a \cdot b) \cdot x$  $\forall a, b \in \mathbb{K}, x \in V$.
		\item \textit{(Scalar Identity)} $1 \cdot x = x$  $\forall x \in V$.
		\item \textit{(Distributivity 1)}  $a \cdot (x + y) = a \cdot x + a \cdot y$  $\forall a \in \mathbb{K}, x, y \in V$.
		\item \textit{(Distributivity 2)} $(a + b) \cdot x = a \cdot x + b \cdot x$  $\forall a, b \in \mathbb{K}, x \in V$.
	\end{enumerate}
} 
\example{
The following are examples of vector spaces.
\begin{itemize}
	\item $\reals^n$, with addition and multiplication as we have defined so far.
	\item $\{0_V\}$, the set containing just the zero vector, is a vector space over any field.
	\item The set of polynomials with degree $3$ or less. Addition and multiplication are defined as $(f+g)(x)=f(x)+g(x)$ and $cf(x)= c\times f(x)$.
	\item $\reals$ is a $\mathbb{Q}$-vector space, where $\mathbb{Q}$ is the set of all rationals (fractions).
\end{itemize}
}
\example{
	The following are non-examples of vector spaces.
	\begin{itemize}
		\item $\phi$, the empty set is not a vector space over any field as it does not have a zero vector.
		\item The set of polynomials with degree $3$ or more (using the addition and multiplication rules defined above).
	\end{itemize}
}
More frequently, we make new vector spaces from taking subsets of existing vector spaces. For instance, there might be a subset of $\reals^n$ that also satisfies all the axioms of being a vector space.

\definition{Subspace}{
	Let $V$ be a $\mathbb{K}$ vector space. We call $W\subseteq V$ a \textbf{subspace} of $V$ if $W$ forms a vector space using the inherited operations $+ : W\times W \to W$ and $\cdot : \mathbb{K}\times W \to W$. 
} 
\begin{remark}
	Because the inherited operations will automatically satisfy the axioms of a vector space, it suffices to show that (1) $W$ is nonempty and (2) $W$ is closed under vector addition and scalar multiplication to confirm that $W$ is a subspace.
\end{remark}
The condition $W$ is nonempty is required because the identity axiom requires a zero vector, which one obtains by multiplying $0$ to an arbitrary vector in the subset.
\example{
\begin{enumerate}
	\item In $\reals$-vector space $\reals^3$, the set of all vectors in the form of $k\vec{i}$ forms a subspace.
	\item Take $\reals$ as a $\mathbb{Q}$-vector space. $\mathbb{Q}\subset\mathbb{R}$ is a subspace. 
	\item Take $\reals$ as a $\mathbb{R}$-vector space. $\mathbb{Q}\subset\mathbb{R}$ is \textbf{not} a subspace because it is not closed under multiplication. 
\end{enumerate}
}
\proposition{
Let $W,U\subseteq V$ be subspaces. The intersection $W\cap U$ is a subspace of $V$.
}
\begin{proof}
	Both $W$ and $U$ contain $0_V$, so the intersection is non-empty. We also have for $v_1,v_2\in W\cap U$,\[
		v_1 + v_2 \in U, \quad v_1+v_2\in W,
	\] so addition is closed in $W\cap U$. Similarly, scalar multiplication closed under $W$ and $U$, so is closed in the intersection.
\end{proof}
\todo 
\exercises
\begin{exerciselist}
	\item a
\end{exerciselist}

